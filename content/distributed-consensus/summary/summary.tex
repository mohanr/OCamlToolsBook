\documentclass[../../../include/open-logic-chapter]{subfiles}

\begin{document}

\olchapter{distcon}{summary}{Summary}


% add commands to create the Tikz font and a degree-symbol
\newcommand{\grad}{$^\circ$}

%to highlight the latex-code
\lstset{
	basicstyle=\small\ttfamily,						% sets the basic style of the code
	language={[LaTeX]TeX},
    texcl=true,  % Enables correct recognition of \usepackage commands
    escapeinside={(*@}{@*)},
	numbersep=5mm, 									% space between line-number and code
	numbers=left, 									% line-numbers on the left side
	numberstyle=\small,								% style of line-numbers
	breaklines=true,								% allows line-breaks
	framexleftmargin=8mm,							% border begins 8mm earlier (include line numbers into box)
	xleftmargin=1.5cm,								% 1.5cm space from the left of text-start
	aboveskip=0.8cm,								% 0.8cm space above listing
	belowskip=0.4cm,								% 0.4cm space below listing
	backgroundcolor=\color{green!7},				% some slight-green background color
	captionpos=b,									% caption should be below (b) the listing
	frame=lrtb,										% adds a frame around the code
	tabsize=4,										% sets default tabsize to 2 spaces
	escapeinside=??,								% escape sequence inside lstlisting
	rulecolor=\color{red},							% bordercolor
	morekeywords={\usetikzlibrary,					% adds more custom keywords
	\node,
	\path,
	\edge},
	keywordstyle=\color[rgb]{0,0,1},				% style for the keywords
    commentstyle=\color[rgb]{0.133,0.545,0.133},	% style for the comments
    stringstyle=\color[rgb]{0.627,0.126,0.941}		% style for the strings
}
% Define a style for C++
\lstdefinestyle{ocamlstyle}{
	basicstyle=\small\ttfamily,
	language=C++,
	numbers=left,
	numberstyle=\small,
	breaklines=true,
	backgroundcolor=\color{gray!10},
	keywordstyle=\color[rgb]{1,0,0},
	commentstyle=\color[rgb]{0,0.5,0},
	stringstyle=\color[rgb]{0,0,1},
	tabsize=4
}
\renewcommand{\lstlistingname}{Code}
\captionsetup[lstlisting]{font={footnotesize},margin=1.5cm,singlelinecheck=false } % removes "Listing 1: "
\definecolor{light-light-gray}{gray}{0.95}

%add a 1 to the authors thanks (instead of a *)
\makeatletter
\let\@fnsymbol\@arabic
\makeatother


	\maketitle
	\begin{abstract}
The Raft consensus Algorithm was desiged by Diego Ongaro and John Ousterhout at Stanford University. Apart from other characteristics
they argue that it is designed for \textbf{Understandability}.


The following primary characteristics are what the Raft authors mention.
\begin{itemize}
\item { Consensus is agreement of shared state}
\item { System is up if majority of servers are up}
\item { Needed for consistent, fault-tolerant storage systems}
\end{itemize}
\end{abstract}\newpage
	\section{Remote Procedure Calls and State Machine}

		\begin{figure}[ht]
			\centering
\begin{tikzpicture}[auto,
            > = Stealth,
every edge quotes/.style = {font=\footnotesize}, % if you like to have smaller edge labels
every edge/.append style = {->, draw=cyan, thick},
every loop/.append style = {<-, looseness = 12},
node distance = 32mm,
 state/.style = {circle, semithick, draw=cyan, text=cyan, minimum size=1.2em},
      rect/.style = {rectangle, draw=cyan, thick, minimum width=2.5cm, minimum height=1.5cm, align=center}
]
\node (A) [state, initial, accepting]   {Follower};
\node (B) [state, below=of A]           {Leader};
\node (C) [state, below=of B]           {Candidate};
\node (A1)  [rect,right=of A,draw,thick,minimum width=2cm,minimum height=2cm] {Passive but expects \\ regular heartbeats};
\node (B1)  [rect,right=of B,draw,thick,minimum width=2cm,minimum height=2cm] {Issues RequestVote RPCs \\ to get elected as leader};
\node (C1)  [rect,right=of C,draw,thick,minimum width=2cm,minimum height=2cm] {Issues AppendVote RPCs: \\Replicates its log\\Hearbeats to maintain leadership};
\path
        (C) edge [bend left,"Discover higher term"]     (A)
        (C) edge [bend left,"Discover higher term"]     (B)
        (A) edge ["Promote"]     (B)
        (B) edge [ "WIn election"]   (C)
        (A) edge [ right]   (A1)
        (B) edge [ right]  (B1)
        (C) edge [ right]  (C1);
    \end{tikzpicture}
	\caption{A machine, accepting the language $L$ with $L=\{0,1\}^*\cdot \{101\}\cdot \{0,1\}^*$.}
		\end{figure}

\section{Leader Election}

			\begin{figure}[ht]
			\centering
\begin{tikzpicture}[auto,
            > = Stealth,
every edge quotes/.style = {font=\footnotesize}, % if you like to have smaller edge labels
every edge/.append style = {->, draw=cyan, thick},
every loop/.append style = {<-, looseness = 12},
node distance = 12mm,
 state/.style = {circle, semithick, draw=cyan, text=cyan, minimum size=1.2em},
      rect/.style = {rectangle, draw=cyan, thick, minimum width=2.5cm, minimum height=1.5cm, align=center}
]
\node (A) [state, initial,initial text=Start up or recovers from crash, accepting,align=center]   {Become \\Candidate};
\node (B) [state, below=of A,align=center]           {CurrentTerm++, \\Vote for self};
\node (C) [state, below=of B,align=center]           {Send RequestVote\\ RPCs};
\node (D) [state, below=of C,align=center]           {Become Leader,\\ Sent heartbeats};
\node (E) [state, below=of D,align=center]           {Become\\ follower};
\path
        (C) edge [bend left=60,"Discover higher term"]     (A)
        (C) edge [bend left=40,"Timeout"]     (B)
        (C) edge [bend right=40,"Votes from majority"]     (D)
        (C) edge [bend left=40, "RPC from leader"]   (E)
    \end{tikzpicture}
	\caption{John Outershout's presentation $L$ with $L=\{0,1\}^*\cdot \{101\}\cdot \{0,1\}^*$.}
	\end{figure}

			\begin{figure}[ht]
			\centering
\begin{tikzpicture}[auto,
            > = Stealth,
every edge quotes/.style = {font=\footnotesize}, % if you like to have smaller edge labels
every edge/.append style = {->, draw=cyan, thick},
every loop/.append style = {<-, looseness = 12},
node distance = 12mm,
 state/.style = {circle, semithick, draw=cyan, text=cyan, minimum size=1.2em},
      rect/.style = {rectangle, draw=cyan, thick, minimum width=2.5cm, minimum height=1.5cm, align=center},
          double state/.style = {circle, semithick, draw=orange, double, double distance=2pt, text=cyan, minimum size=1.2em}
        ]

\def\radius{3}

% Define nodes in a circular layout
\node[double state] (N1) at ({1*360/5}:\radius) {Node 1}; % Double circle for Node 1
\node[double state] (N2) at ({2*360/5}:\radius) {Node 2}; % Double circle for Node 2
\node[double state] (N3) at ({3*360/5}:\radius) {Node 3}; % Double circle for Node 3
\node[state] (N4) at ({4*360/5}:\radius) {Node 4}; % Normal node
\node[state] (N5) at ({5*360/5}:\radius) {Node 5}; % Normal node

\path (N5) edge[bend right=20] (N1);
\path (N1) edge[bend right=20] (N2);
\path (N2) edge[bend right=20] (N3);
\path (N3) edge[bend right=20] (N4);
\path (N4) edge[bend right=20] (N5);
\newcommand\DrawControl[3]{
  node[#2,circle,fill=#2,inner sep=2pt,label={above:$#1$},label={[black]below:{\footnotesize#3}}] at #1 {}
}
    \end{tikzpicture}
	\caption{John Outershout's presentation $L$ with $L=\{0,1\}^*\cdot \{101\}\cdot \{0,1\}^*$.}
		\end{figure}

\section{The Term}
	\subsection{Create an environment}
		A term is a value that is sent with every RPC and received in every response. It is used to identify obsolete information \textit{(e.g)} If a peer has a later term, the term is updated and the status is reverted to \textit{Follower}.
		Every server maintains its own term and so there is no-\textit{Global view}.

		To draw graphs, uses \texttt{node}s and \texttt{path}s. Nodes are the states/boxes/circles of a graph, while the path describes one or more line between them.

		To getting stared with drawing in , we must create an environment:\\
		%% \begin{minipage}{\linewidth}

			\begin{lstlisting}[caption={Create a simple enviroment.}]
\begin{tikzpicture}
...
\end{tikzpicture}
			\end{lstlisting}
		%% \end{minipage}

	\subsection{Parameter of an environment}
		For this environment, there are different parameter, we can choose from. For instance:\\
		%% \begin{minipage}{\linewidth}

			\begin{lstlisting}[style=ocamlstyle,caption={A example with parameter in a environment.}]
  let get_state = function
    | `Leader -> "leader."
    | `Follower -> "follower."
    | `Candidate -> "candidate."
    | `Dead -> " dead."
			\end{lstlisting}

		%% \end{minipage}

		Here's a short explanation of these parameter:\\
		\begin{itemize}
			\item \texttt{RequestVote} : Solicits votes from other members of the cluster
			\item \texttt{AppendEntries} : Replicates the log and can\\
also server as a heartbeat
		\end{itemize}
		More detailed information on the different parameter now.
	\subsection{Basic and miscellaneous parameter}
		Here just a few basic parameter.
		\begin{itemize}
			\item \texttt{-} : All connections are normal lines
			\item \texttt{->} : All connections are arrows
			\item \texttt{>= <Option>} : Specifies the type of the arrow head
			\item \texttt{shorten >= x} : Distance arrow head -- Node (e.g. 5pt, 1cm, ...)
		\end{itemize}
	\subsection{Color of graphs}
		You can change the color of the graph with the paremeter. To do that, just use the \texttt{color=<color>} option.\\
		%% \begin{minipage}{\linewidth}

			\begin{lstlisting}[caption={Choosing a slight red color (60\% red, 40\% white/background)}]
\begin{tikzpicture}[->,
	thin,
	color=red!60]
...
\end{tikzpicture}
			\end{lstlisting}
		%% \end{minipage}

	\section{Arrows and lines}
	\subsection{Thickness of lines and arrows}
		There're different thickness options available for arrows and lines (top: very thin, bottom: thickest):
		\begin{itemize}
			\item \texttt{\textbackslash ultra thin}
			\item \texttt{\textbackslash very thin}
			\item \texttt{\textbackslash thin}
			\item \texttt{\textbackslash semithick}
			\item \texttt{\textbackslash thick}
			\item \texttt{\textbackslash very thick}
			\item \texttt{\textbackslash ultra thick}
		\end{itemize}
	\subsection{Arrow heads}
		To use arrows properly, add the  library \texttt{arrows} to your collection:\\
		%% \begin{minipage}{\linewidth}

			\begin{lstlisting}[caption={The library \texttt{arrows} is used to draw arrows.}]
\usetikzlibrary{arrows}
			\end{lstlisting}
		%% \end{minipage}

		There're many different - party strange - types of arrows.
		The type of an arrow can also be specified at the beginning of an option.\\
		%% \begin{minipage}{\linewidth}

			\begin{lstlisting}[caption={Specification of an arrow type.}]
\begin{tikzpicture}[->,
	>=<arrow-type>]
...
\end{tikzpicture}
			\end{lstlisting}
		%% \end{minipage}

		Here's a small - for state-machines interesting - selection of different arrow types:
		\begin{itemize}
			\item no value : \tikz\draw[->,semithick] (0,0) -- (1,0) node [pos=0.2, right] {};
			\item \texttt{latex} : \tikz\draw[->,>=latex,semithick] (0,0) -- (1,0) node [pos=0.2, right] {};
			\item \texttt{latex'} : \tikz\draw[->,>=latex',semithick] (0,0) -- (1,0) node [pos=0.2, right] {};
			\item \texttt{stealth} : \tikz\draw[->,>=stealth,semithick] (0,0) -- (1,0) node [pos=0.2, right] {};
			\item \texttt{stealth'} : \tikz\draw[->,>=stealth',semithick] (0,0) -- (1,0) node [pos=0.2, right] {};
			\item \texttt{triangle <angle>} und \texttt{open triangle <angle>} : \\
				\hspace*{10pt} With 45\grad : \tikz\draw[->,>=triangle 45,semithick] (0,0) -- (1,0) node [pos=0.2, right] {}; \tikz\draw[->,>=open triangle 45,semithick] (2,0) -- (3,0) node [pos=0.2, right] {};\\
				\hspace*{10pt} With 60\grad : \tikz\draw[->,>=triangle 60,semithick] (0,0) -- (1,0) node [pos=0.2, right] {}; \tikz\draw[->,>=open triangle 60,semithick] (2,0) -- (3,0) node [pos=0.2, right] {};\\
				\hspace*{10pt} With 90\grad : \tikz\draw[->,>=triangle 90,semithick] (0,0) -- (1,0) node [pos=0.2, right] {}; \tikz\draw[->,>=open triangle 90,semithick] (2,0) -- (3,0) node [pos=0.2, right] {};
			\item \texttt{angle <angle>} \\
				\hspace*{10pt} With 45\grad : \tikz\draw[->,>=angle 45,semithick] (0,0) -- (1,0) node [pos=0.2, right] {};\\
				\hspace*{10pt} With 60\grad : \tikz\draw[->,>=angle 60,semithick] (0,0) -- (1,0) node [pos=0.2, right] {};\\
				\hspace*{10pt} With 90\grad : \tikz\draw[->,>=angle 90,semithick] (0,0) -- (1,0) node [pos=0.2, right] {};
		\end{itemize}
		The option \texttt{'} with the \texttt{latex'} creates a hyperbolic sharpened triangle as the arrow head. \texttt{stealth'} makes the edges more round than sharp.

		Furthermore can the option \texttt{->>} be used to create a double arrow with two heads (e.g. with the \texttt{stealth'} option: \tikz\draw[->>,>=stealth',semithick] (0,0) -- (1,0) node [pos=0.2, right] {}; ). This is possible with all kinds of arrows.
	\section{Nodes}
		Nodes are used to create states for our state-machine(s). Thanks to the library \texttt{automata} there are lots of different predefined options. To use these options, just integrate the library \texttt{automata}\\

		To position the created node very quickly and easily, you should use the library \texttt{positioning}:\\
		%% \begin{minipage}{\linewidth}

			\begin{lstlisting}[caption={Using the library \texttt{automata} and \texttt{positioning}}]


\usetikzlibrary{automata}
\usetikzlibrary{positioning}
			\end{lstlisting}
		%% \end{minipage}

	\subsection{Create nodes}
		Now we can create the first nodes with \texttt{\textbackslash node}. Wherein it should be noted that there's a specific syntax:\\
		%% \begin{minipage}{\linewidth}

			\begin{lstlisting}[caption={Syntax of the \texttt{\textbackslash node} command.}]

\node[<optionen>] (name) [<optional: position>] {text};
			\end{lstlisting}
		%% \end{minipage}\\

		When no position is specified, the node will be placed at the origin (normally at the very left side). The parameter for the positioning will be discussed later.

		Important to be mentioned is the opportunity to write the displayed text of a node in math-mode, which allows indices like $z_0$ or something.
		\subsubsection{Example machine without connections}
		A state-machine with the three states $z_0$, $z_1$ and $z_2$ can be look like this:\\
		%% \begin{minipage}{\linewidth}

			\begin{lstlisting}[mathescape,caption={Drawing of a very simple machine.}]
\begin{tikzpicture}[->,
	>=stealth',
	semithick]

	\node[state,initial]	(0)              {$z_0$};
	\node[state]			(1) [right of=0] {$z_1$};
	\node[state,accepting]	(2) [right of=1] {$z_2$};
\end{tikzpicture}
			\end{lstlisting}
		%% \end{minipage}

		Ans here's the result:\\\\
		\begin{figure}[ht]
			\centering
			\begin{tikzpicture}[->,
				>=stealth',
				semithick]

				\node[state,initial]	(0)              {$z_0$};
				\node[state]			(1) [right of=0] {$z_1$};
				\node[state,accepting]	(2) [right of=1] {$z_2$};
			\end{tikzpicture}
			\caption{A simple machine with three states.}
		\end{figure}\\\\
		You can passing over parameters to the \texttt{\textbackslash node} command from the \texttt{automata} library as well.
	\subsection{Options for states}
		Here's a list of some useful options for states (\texttt{\textbackslash node}s):\\
		\begin{itemize}
			\item \texttt{state} : Has to be specified to create a state
			\item \texttt{initial} : Specifies a start-state\\
				\hspace*{10pt} Creates an arrow with \textit{start}-Label
			\item \texttt{accepting} : Indicates the final state of the machine\\
				\hspace*{10pt} Draws a double circle around the state
			\item \texttt{with output} : Creates a state with output\\
				\hspace*{10pt} More information \href{mailto:mail@hauke-stieler.de}{if desired}
		\end{itemize}
	\subsection{Positioning of the states}
		The position of the states can easily be specified by using the \texttt{positioning} library.
		\subsubsection{Options for positioning}
		For the positioning are - amongst others - the following options available. These options can also be combined:\\
		\begin{itemize}
			\item \texttt{right, left, above, below} : Next to the node (in given direction)
			\item \texttt{of = <Node>} : Specifies the node use mean by using \texttt{right, left, ...}
			\item \texttt{and} : When you want to use several distances, use \texttt{and}\\
			Ex.: \texttt{[...=2cm \underline{and} 1cm of A]}
		\end{itemize}
		\subsubsection{Automatic distance}
		It should be considered that there's a specific syntax for the \textbf{automatic distance} of nodes.\\
		%% \begin{minipage}{\linewidth}

			\begin{lstlisting}[caption={Syntax for the positioning of nodes.}]
[<position> of = <reference-node>]
			\end{lstlisting}
		%% \end{minipage}\\

		The value of the automatic distance can be defined by using \texttt{node distance=...} with the normal \LaTeX units (\texttt{cm, em, pt, ...}).
		So for example:\\
		%% \begin{minipage}{\linewidth}

			\begin{lstlisting}[mathescape,caption={Automatic positioning of a node to the right of A.}]
[right of = A]
			\end{lstlisting}
		%% \end{minipage}\\

		This placed a node to the right of a node A, while the distance defined by \texttt{node distance=...} is used.
		\subsubsection{Manual distance}
		To use a \textbf{manual distance}, which can be specified by using the well known \LaTeX units \texttt{cm, em, pt, ...}, you have to consider the syntax:\\
		%% \begin{minipage}{\linewidth}

			\begin{lstlisting}[caption={Syntax for the manual positioning of nodes.}]
[<position> = <distance> of <reference-node>]
			\end{lstlisting}
		%% \end{minipage}\\

		For example:\\
		%% \begin{minipage}{\linewidth}

			\begin{lstlisting}[mathescape,caption={Manual positioning of a node right of A}]
[right = 2cm of A]
			\end{lstlisting}
		%% \end{minipage}\\

		This placed the one node 2cm to the right of node A.
		\subsubsection{Example for the positioning of nodes}
		Given are three nodes: $a$, $b$ and $c$.
		\begin{itemize}
			\item Node $b$ should be placed right and above (2cm right, 1cm above) the Node $a$
			\item Node $c$ should be placed automatically right and below of node $a$
		\end{itemize}
		Therefore you can skillfully combine the options:\\
		%% \begin{minipage}{\linewidth}

			\begin{lstlisting}[mathescape,caption={Combination of options for the positioning.}]
\begin{tikzpicture}[->,
	>=stealth',
	semithick,
	node distance=2cm]

\node [state] (a)                                {$a$};
\node [state] (b) [above right=1cm and 2cm of a] {$b$};
\node [state] (c) [below right of = a]           {$c$};

\end{tikzpicture}
			\end{lstlisting}
		%% \end{minipage}\\

		If you choose a nice value for the distance (e.g. \texttt{node distance=2cm}), it can look pretty fancy:\\\\
		\begin{figure}[ht]
			\centering
			\begin{tikzpicture}[->,
				>=stealth',
				semithick,
				node distance=2cm]

				\node [state] (a)                                {$a$};
				\node [state] (b) [above right=1cm and 2cm of a] {$b$};
				\node [state] (c) [below right of = a]           {$c$};
			\end{tikzpicture}
			\caption{Three different placed nodes.}
		\end{figure}\\\\
	\section{Nodes verbinden}
	To not only have a bunch of states but a real state-machine, you have to connect these states. There's a command called \texttt{\textbackslash path} to create those connections.
	\subsection{Create arrows}
		\texttt{\textbackslash path} has a syntax as well:\\
		%% \begin{minipage}{\linewidth}

			\begin{lstlisting}[mathescape,caption={Syntax of the \texttt{\textbackslash path} command.}]
\path (<from-node>) edge [<options>] node {text} (<to-node>);
			\end{lstlisting}
		%% \end{minipage}\\

		Important to be noted is that you need to write \texttt{\textbackslash path} just once (see example below).\\
		What kind of options exist, is coming soon, but at first we'll see an example of this:\\
		%% \begin{minipage}{\linewidth}

			\begin{lstlisting}[mathescape,caption={The well known machine from before, but this time with some connections.}]
\begin{tikzpicture}[->,
	>=stealth',
	semithick,
	node distance=2cm]

\node [state,initial] (a)                        {$a$};
\node [state] (b) [above right=1cm and 2cm of a] {$b$};
\node [state,accepting] (c) [below right of = a] {$c$};

\path (a) edge node {0} (b)
          edge node {1} (c)
      (c) edge node {2} (b);

\end{tikzpicture}
			\end{lstlisting}
		%% \end{minipage}\\

		And here's the belonging machine\\
		\begin{figure}[ht]
			\centering
			\begin{tikzpicture}[->,
				>=stealth',
				semithick,
				node distance=2cm]

				\node [state,initial] (a)                        {$a$};
				\node [state] (b) [above right=1cm and 2cm of a] {$b$};
				\node [state,accepting] (c) [below right of = a] {$c$};

				\path (a) edge node {0} (b)
				          edge node {1} (c)
				      (c) edge node {2} (b);

			\end{tikzpicture}
			\caption{Three states with come connections (\textbf{without} options.}
		\end{figure}
	\subsection{Options for the \texttt{\textbackslash path} command}
		You can choose on different options as well:
		\begin{itemize}
			\item \texttt{right, left, above, below} : Specifies the position of the text
			\item \texttt{bend <right, left>} : Creates a bended edge. The direction of the bending is specified from the perspective of the arrow
			\item \texttt{bend <right, left> = <angle>} : The \texttt{angle} describes the strength of the bending
			\item \texttt{loop <right, left, above, below>} : Creates a loop above, below, left or right of the state
		\end{itemize}
	\section{Examples}
		Because the most important aspects as been discussed, you'll find a few examples down here.
	\subsection{Example 1}

		\begin{figure}[ht]
			\centering
			\begin{tikzpicture}[->,>=stealth',
				shorten >=5pt,
				node distance=4.5cm,
				semithick]

				\node[initial,state]   (R)              {Follower};
				\node[state]           (S) [right of=R] {Candidate};
				\node[state,accepting]           (T) [right of=S] {Leader};

				\path 	(R)
				            edge [bend left=50,,above]            node {timeout,start election} (S)
						(S) edge [bend left=60,below]            node {discover curent leader or higher term} (R)
						(S) edge [loop above]            node {timeout, new election} (S)
						(S) edge [bend left,above]            node {receive votes from majority of servers} (T)

						(T) edge [bend left=20,below]  node {discover server with higher term} (R)
			            ;
			\end{tikzpicture}\\
			\caption{A machine, accepting the language $L$ with $L=\{0,1\}^*\cdot \{101\}\cdot \{0,1\}^*$.}
		\end{figure}
	\subsection{Example 2}
		%% \end{minipage}\\
