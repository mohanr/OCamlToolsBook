\documentclass[../../../include/open-logic-chapter]{subfiles}

\begin{document}

\olchapter{distributed-consensus}{raft}{RAFT Distributed consensus protocol}


% add commands to create the Tikz font and a degree-symbol
\newcommand{\grad}{$^\circ$}

%to highlight the latex-code
\lstset{
	basicstyle=\small\ttfamily,						% sets the basic style of the code
	language={[LaTeX]TeX},
    texcl=true,  % Enables correct recognition of \usepackage commands
    escapeinside={(*@}{@*)},
	numbersep=5mm, 									% space between line-number and code
	numbers=left, 									% line-numbers on the left side
	numberstyle=\small,								% style of line-numbers
	breaklines=true,								% allows line-breaks
	framexleftmargin=8mm,							% border begins 8mm earlier (include line numbers into box)
	xleftmargin=1.5cm,								% 1.5cm space from the left of text-start
	aboveskip=0.8cm,								% 0.8cm space above listing
	belowskip=0.4cm,								% 0.4cm space below listing
	backgroundcolor=\color{green!7},				% some slight-green background color
	captionpos=b,									% caption should be below (b) the listing
	frame=lrtb,										% adds a frame around the code
	tabsize=4,										% sets default tabsize to 2 spaces
	escapeinside=??,								% escape sequence inside lstlisting
	rulecolor=\color{red},							% bordercolor
	morekeywords={\usetikzlibrary,					% adds more custom keywords
	\node,
	\path,
	\edge},
	keywordstyle=\color[rgb]{0,0,1},				% style for the keywords
    commentstyle=\color[rgb]{0.133,0.545,0.133},	% style for the comments
    stringstyle=\color[rgb]{0.627,0.126,0.941}		% style for the strings
}
% Define a style for C++
\lstdefinestyle{ocamlstyle}{
	basicstyle=\small\ttfamily,
	language=C++,
	numbers=left,
	numberstyle=\small,
	breaklines=true,
	backgroundcolor=\color{gray!10},
	keywordstyle=\color[rgb]{1,0,0},
	commentstyle=\color[rgb]{0,0.5,0},
	stringstyle=\color[rgb]{0,0,1},
	tabsize=4
}
\renewcommand{\lstlistingname}{Code}
\captionsetup[lstlisting]{font={footnotesize},margin=1.5cm,singlelinecheck=false } % removes "Listing 1: "
\definecolor{light-light-gray}{gray}{0.95}

%add a 1 to the authors thanks (instead of a *)
\makeatletter
\let\@fnsymbol\@arabic
\makeatother


	\begin{abstract}
The Raft consensus Algorithm was desiged by Diego Ongaro and John Ousterhout at Stanford University. Apart from other characteristics
they argue that it is designed for \textbf{Understandability}.


The following primary characteristics are what the Raft authors mention.
\begin{itemize}
\item { Consensus is agreement of shared state}
\item { System is up if majority of servers are up}
\item { Needed for consistent, fault-tolerant storage systems}
\end{itemize}
\end{abstract}\newpage
	\section{Remote Procedure Calls and State Machine}

		\begin{figure}[ht]
			\centering
\begin{tikzpicture}[auto,
            > = Stealth,
every edge quotes/.style = {font=\footnotesize}, % if you like to have smaller edge labels
every edge/.append style = {->, draw=cyan, thick},
every loop/.append style = {<-, looseness = 12},
node distance = 32mm,
 state/.style = {circle, semithick, draw=cyan, text=cyan, minimum size=1.2em},
      rect/.style = {rectangle, draw=cyan, thick, minimum width=2.5cm, minimum height=1.5cm, align=center}
]
\node (A) [state, initial, accepting]   {Follower};
\node (B) [state, below=of A]           {Leader};
\node (C) [state, below=of B]           {Candidate};
\node (A1)  [rect,right=of A,draw,thick,minimum width=2cm,minimum height=2cm] {Passive but expects \\ regular heartbeats};
\node (B1)  [rect,right=of B,draw,thick,minimum width=2cm,minimum height=2cm] {Issues RequestVote RPCs \\ to get elected as leader};
\node (C1)  [rect,right=of C,draw,thick,minimum width=2cm,minimum height=2cm] {Issues AppendVote RPCs: \\Replicates its log\\Hearbeats to maintain leadership};
\path
        (C) edge [bend left,"Discover higher term"]     (A)
        (C) edge [bend left,"Discover higher term"]     (B)
        (A) edge ["Promote"]     (B)
        (B) edge [ "WIn election"]   (C)
        (A) edge [ right]   (A1)
        (B) edge [ right]  (B1)
        (C) edge [ right]  (C1);
    \end{tikzpicture}
	\caption{}
		\end{figure}

\section{Leader Election}

			\begin{figure}[ht]
			\centering
\begin{tikzpicture}[auto,
            > = Stealth,
every edge quotes/.style = {font=\footnotesize}, % if you like to have smaller edge labels
every edge/.append style = {->, draw=cyan, thick},
every loop/.append style = {<-, looseness = 12},
node distance = 12mm,
 state/.style = {circle, semithick, draw=cyan, text=cyan, minimum size=1.2em},
      rect/.style = {rectangle, draw=cyan, thick, minimum width=2.5cm, minimum height=1.5cm, align=center}
]
\node (A) [state, initial,initial text=Start up or recovers from crash, accepting,align=center]   {Become \\Candidate};
\node (B) [state, below=of A,align=center]           {CurrentTerm++, \\Vote for self};
\node (C) [state, below=of B,align=center]           {Send RequestVote\\ RPCs};
\node (D) [state, below=of C,align=center]           {Become Leader,\\ Sent heartbeats};
\node (E) [state, below=of D,align=center]           {Become\\ follower};
\path
        (C) edge [bend left=60,"Discover higher term"]     (A)
        (C) edge [bend left=40,"Timeout"]     (B)
        (C) edge [bend right=40,"Votes from majority"]     (D)
        (C) edge [bend left=40, "RPC from leader"]   (E)
    \end{tikzpicture}
	\caption{John Outershout's presentation.}
	\end{figure}

			\begin{figure}[ht]
			\centering
\begin{tikzpicture}[auto,
            > = Stealth,
every edge quotes/.style = {font=\footnotesize}, % if you like to have smaller edge labels
every edge/.append style = {->, draw=cyan, thick},
every loop/.append style = {<-, looseness = 12},
node distance = 12mm,
 state/.style = {circle, semithick, draw=cyan, text=cyan, minimum size=1.2em},
      rect/.style = {rectangle, draw=cyan, thick, minimum width=2.5cm, minimum height=1.5cm, align=center},
          double state/.style = {circle, semithick, draw=orange, double, double distance=2pt, text=cyan, minimum size=1.2em}
        ]

\def\radius{3}

% Define nodes in a circular layout
\node[double state] (N1) at ({1*360/5}:\radius) {Node 1}; % Double circle for Node 1
\node[double state] (N2) at ({2*360/5}:\radius) {Node 2}; % Double circle for Node 2
\node[double state] (N3) at ({3*360/5}:\radius) {Node 3}; % Double circle for Node 3
\node[state] (N4) at ({4*360/5}:\radius) {Node 4}; % Normal node
\node[state] (N5) at ({5*360/5}:\radius) {Node 5}; % Normal node

\path (N5) edge[bend right=20] (N1);
\path (N1) edge[bend right=20] (N2);
\path (N2) edge[bend right=20] (N3);
\path (N3) edge[bend right=20] (N4);
\path (N4) edge[bend right=20] (N5);
\newcommand\DrawControl[3]{
  node[#2,circle,fill=#2,inner sep=2pt,label={above:$#1$},label={[black]below:{\footnotesize#3}}] at #1 {}
}
    \end{tikzpicture}
	\caption{John Outershout's presentation}
		\end{figure}

\section{The Term}
		A term is a value that is sent with every RPC and received in every response. It is used to identify obsolete information \textit{(e.g)} If a peer has a later term, the term is updated and the status is reverted to \textit{Follower}.
		Every server maintains its own term and so there is no-\textit{Global view}.


			\begin{lstlisting}[style=ocamlstyle,caption={A example with parameter in a environment.}]
  let get_state = function
    | `Leader -> "leader."
    | `Follower -> "follower."
    | `Candidate -> "candidate."
    | `Dead -> " dead."
			\end{lstlisting}

		%% \end{minipage}

		\begin{itemize}
			\item \texttt{RequestVote} : Solicits votes from other members of the cluster
			\item \texttt{AppendEntries} : Replicates the log and can\\
also server as a heartbeat
		\end{itemize}
		\begin{figure}[ht]
			\centering
			\begin{tikzpicture}[->,>=stealth',
				shorten >=5pt,
				node distance=4.5cm,
				semithick]

				\node[initial,state]   (R)              {Follower};
				\node[state]           (S) [right of=R] {Candidate};
				\node[state,accepting]           (T) [right of=S] {Leader};

				\path 	(R)
				            edge [bend left=50,,above]            node {timeout,start election} (S)
						(S) edge [bend left=60,below]            node {discover curent leader or higher term} (R)
						(S) edge [loop above]            node {timeout, new election} (S)
						(S) edge [bend left,above]            node {receive votes from majority of servers} (T)

						(T) edge [bend left=20,below]  node {discover server with higher term} (R)
			            ;
			\end{tikzpicture}\\
			\caption{}
		\end{figure}
		%% \end{minipage}\\

\OLEndChapterHook
