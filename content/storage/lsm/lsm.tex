\documentclass[../../../include/open-logic-chapter]{subfiles}

\begin{document}

\olchapter{sge}{lsm}{Log Structured Merge}

\begin{abstract}
\end{abstract}
\newpage

	\section{ Construction of LSM Tree}
	\section{ SkipLists}
	\section{ Bloom Filters}

	\subsection{ Development of a hash function }

    I have settled on \textit{Murmurhash} to test my bloom filter.
    The qualities of the hash function and its suitabality are not investigated at this time.

I decided to construct the little-endian order and I wanted proper
\textit{unsigned ints.}.

			\begin{lstlisting}[style=ocamlstyle,caption={Create little-endian order by shifting 8-bit sequences.}]
  (* k := Bytes.get_int32_le chunks i;
   TODO Bytes.get_int32_le didn't give the right result. *)

  k := Int32.logor
       (Int32.logor
         (Int32.of_int (Bytes.get_uint8 chunks (i * 4)))
         (Int32.shift_left (Int32.of_int (Bytes.get_uint8 chunks ((i * 4)+ 1))) 8))
       (Int32.logor
         (Int32.shift_left (Int32.of_int (Bytes.get_uint8 chunks ((i * 4)+ 2))) 16)
         (Int32.shift_left (Int32.of_int (Bytes.get_uint8 chunks ((i * 4)+ 3))) 24));
			\end{lstlisting}

The following is the entire function ported from its \textit{C} source.

			\begin{lstlisting}[style=ocamlstyle,caption={The entire murmurhash function.}]
open Int32

let murmurhash chunks len seed =
  let c1 =  0xcc9e2d51l in
  let c2 =  0x1b873593l in
  let r1:int32 = (of_int 15) in
  let r2:int32 = (of_int 13) in
  let m = (of_int 5) in
  let n =  (of_string "0xe6546b64") in
  let h = ref zero in
  let k = ref zero in
  let l = div len (of_int 4) in
  h := seed;

 (* Printf.eprintf " %ld"  l; *)

 for i = 0 to (to_int l) - 1 do
  (* k := Bytes.get_int32_le chunks i; *)

  k := logor
       (logor
         (of_int (Bytes.get_uint8 chunks (i * 4)))
         (shift_left (of_int (Bytes.get_uint8 chunks ((i * 4)+ 1))) 8))
       (logor
         (shift_left (of_int (Bytes.get_uint8 chunks ((i * 4)+ 2))) 16)
         (shift_left (of_int (Bytes.get_uint8 chunks ((i * 4)+ 3))) 24));

  k := mul !k c1 ;
  k := logor (shift_left !k  (to_int r1))  (shift_right_logical !k  (Int32.to_int (Int32.sub (Int32.of_int 32)  r1)));
  k := mul !k c2;

  h := logxor !h !k;
  h := logor (shift_left !h  (to_int r2))  (shift_right_logical !h  (Int32.to_int (Int32.sub (Int32.of_int 32)  r2)));
  h := add ( mul !h m)  n;

  done;

  let k = ref (of_int 0) in
let tail = to_int (mul l 4l) in
let l = (to_int len) - tail in

if l >= 3 then k := logxor !k (shift_left (of_int (Bytes.get_uint8 chunks (tail + 2))) 16);
if l >= 2 then k := logxor !k (shift_left (of_int (Bytes.get_uint8 chunks (tail + 1))) 8);
if l >= 1 then begin
  k := logxor !k (of_int (Bytes.get_uint8 chunks tail));

      k :=  mul !k c1;
      k := logxor (shift_left  !k (to_int r1))
          (shift_right_logical !k  (to_int (sub (of_int 32)  r1)));
      k :=  mul !k c2;
      h := logxor !h !k;
  end;

  h := logxor !h  len;

  h := logxor !h (shift_right_logical !h 16);
  h := mul !h  (of_string "0x85ebca6b");
  h := logxor !h (shift_right_logical !h 13);
  h := mul !h  (of_string "0xc2b2ae35");
  h := logxor !h (shift_right_logical !h 16);

  !h
	\end{lstlisting}

	\subsection{ Some basic tests}

 The observant reader may have noted that there are two primary
 sections in the hash function shown above.


\begin{itemize}
\item { The first part of the code that handles 4-byte sequences}
\item { The remaining part that handles sequences less than 4-bytes}
\end{itemize}

Both parts ar exercised in the test.


			\begin{lstlisting}[style=ocamlstyle,caption={Tests.}]
 let test str seed =
  let len =  (Int32.of_int (String.length str)) in
  murmurhash (String.to_bytes str) len seed

  let%expect_test _=

  let seed = 0 in
  let hash = test "asdfqwerty" (Int32.of_int seed) in
  let hex = Printf.sprintf "%lx" hash in
  print_endline hex;
  [%expect {| a3cfe04b |}];

  let seed = 0 in
  let hash = test "hey" (Int32.of_int seed) in
  let hex = Printf.sprintf "%lx" hash in
  print_endline hex;

  [%expect {|
    12f94418
    |}];


  let hash = test "dude" (Int32.of_int seed) in
  let hex = Printf.sprintf "%lx" hash in
  print_endline hex;
  [%expect {| ef0487f3 |}];
  let hash = test "Hello2" (Int32.of_int seed) in
  let hex = Printf.sprintf "%lx" hash in
  print_endline hex;
  [%expect {| e5ce223e |}]
	\end{lstlisting}

	\section{ MemTable}
	\section{ Sorted String Table}
	\section{ Summary }

\OLEndChapterHook

\end{document}
